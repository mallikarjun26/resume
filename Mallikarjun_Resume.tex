%%%%%%%%%%%%%%%%%%%%%%%%%%%%%%%%%%%%%%%%%
% Medium Length Professional CV
% LaTeX Template
% Version 2.0 (8/5/13)
%
% This template has been downloaded from:
% http://www.LaTeXTemplates.com
%
% Original author:
% Trey Hunner (http://www.treyhunner.com/)
%
% Important note:
% This template requires the resume.cls file to be in the same directory as the
% .tex file. The resume.cls file provides the resume style used for structuring the
% document.
%
%%%%%%%%%%%%%%%%%%%%%%%%%%%%%%%%%%%%%%%%%

%----------------------------------------------------------------------------------------
%	PACKAGES AND OTHER DOCUMENT CONFIGURATIONS
%----------------------------------------------------------------------------------------

\documentclass{resume} % Use the custom resume.cls style

\usepackage[left=0.75in,top=0.6in,right=0.75in,bottom=0.6in]{geometry} % Document margins
\usepackage[utf8]{inputenc}

\name{Mallikarjun B R} % Your name
\address{No 156, NBH Hostel, IIIT-Hyderabad \\ Telangana - 500032} % Your address
\address{(91)~$\cdot$~9964421652 \\ mallik.jeevan@gmail.com} % Your phone number and email

\begin{document}

%----------------------------------------------------------------------------------------
%	EDUCATION SECTION
%----------------------------------------------------------------------------------------

\begin{rSection}{Education}

{\bf International Institute of Information Technology (IIIT-H)} \hfill {\em December 2013 - Present} \\ 
MS by Research (Center for Visual Information Technology) in Computer Science Engineering \smallskip \\
Courses: Statistical Methods in AI, Machine Learning(Audit), Optimization Methods, Computer Vision, Information Retrieval \& Extraction, Digital Image Processing, Intro to Robotics. \\
Overall GPA: 8.66  

{\bf Rashtreeya Vidyalaya College of Engineering (RVCE), Bangalore} \hfill {\em June 2011} \\ 
B.E. in Electronics \& Communication Engineering \smallskip \\
Courses: Artificial Neural Networks, Signals \& Systems. \\
Overall GPA: 9.35
\end{rSection}

%----------------------------------------------------------------------------------------
%	WORK EXPERIENCE SECTION
%----------------------------------------------------------------------------------------

\begin{rSection}{Experience}

\begin{rSubsection}{Juniper Networks}{July 2011 - January 2014}{ASIC Engineer 2, Silicon and Systems Technology}{Bangalore, Karnataka}
\item Part of ASIC development team, which delivered two generations of networking ASICs working on 28nm technology. These ASICs are currently being used in products such as QFabric and M \& T-Series routers. 
\item Modelled a block called TOKEN which does the dynamic arbitration of packets across 1024x1024 Packet Forwarding Engines(PFE) with four levels of priority queuing for each PFE, for design verification.
\item Part of Full-Chip verification team. Also contributed towards Gate-Level simulation and Functional Coverage.
\end{rSubsection}
\end{rSection}

%----------------------------------------------------------------------------------------
%  PROJECTS	
%----------------------------------------------------------------------------------------

\begin{rSection}{PROJECTS}

\begin{rSubsection}{Face fiducial detection by consensus of exemplars}{}{}{}
\item Research project for thesis.
\end{rSubsection}

\begin{rSubsection}{Face video synthesis}{}{}{}
\item Face space is represented (fairly) densely using a person\textsc{\char13}s faces extracted from a video. 
\item Space is represented as a graph with face as nodes and edge weights based on pose and features from fiducial points.
\item Various traversals in the graph explored to synthesize interesting videos.
\item 3D video tensor factorization to extract identity and expression of an individual.
\item Expression factor is introduced to another individual\textsc{\char13}s identity to obtain similar change in expression.
\end{rSubsection}

\begin{rSubsection}{Sign language detection using CNN}{}{}{}
\item Classification of English characters represented as hand signs in images.
\item Convolutional Neural Networks(CNN) has been used to train the classes. 
\item Various configurations with data augmentation, dropout and non-linear functions have been experi-
mented.
\item Accuracy of 60\% achieved as compared to 29\% of baseline model based on Bag of Visual Words.
\end{rSubsection}

\clearpage
\begin{rSubsection}{Document Layout Analysis}{}{}{}
\item Segmentation of document into semantically meaningful blocks using morphological processes.
\item Each segment is categorized into text, table, heading and figure based on connected component analysis, Hough Transform. 
\end{rSubsection}

\begin{rSubsection}{Lead character recognition in a movie}{}{}{}
\item Extracted all faces from the video using haar features.
\item LBP feature vector is constructed for each of the extracted faces.
\item Clustering the feature points into clusters using k-means algorithm.
\item Most similar faces with respect to top 2 clusters(based on the number of points in that cluster) are selected as the lead characters.
\end{rSubsection}

\begin{rSubsection}{Search Engine for Wikipedia}{}{}{}
\item Preprocessed Wikipedia corpus of 44GB. 
\item Built inverted index for the corpus.
\item Search engine based on tf-idf model was developed.
\end{rSubsection}

\begin{rSubsection}{Twitter Entity Disambiguation}{}{}{}
\item Entity disambiguation of tweets. 61 different entities and corresponding tweets as the data set.  
\item Various features extracted considering labelled data set, entity’s home and Wikipedia pages, WordNet etc.
\item Supervised Machine Learning (Support Vector Machines) used to train and test the model.
\item Achieved an accuracy of 40\%-90\% across entities.
\end{rSubsection}

\begin{rSubsection}{3 Link Manipulator(Robot) path planning using Rapidly Exploring Random Tree}{}{}{}
\item Developed kinematic model of the robot. 
\item Exploring configuration space with obstacles using Rapidly Exploring Random Trees(RRT).
\item Find the path from initial to final position avoiding obstacles. 
\end{rSubsection}

\begin{rSubsection}{Mobile Controlled Robot}{}{}{}
\item Robot navigation control using mobile network.  
\item Communication signal carries DTMF (Dual Tone Multiple Frequency), when a key is pressed.
\item Signal is decoded to find the key pressed. Navigation is performed based on the input.
\end{rSubsection}

\end{rSection}


%----------------------------------------------------------------------------------------
%	TECHNICAL STRENGTHS SECTION
%----------------------------------------------------------------------------------------

\begin{rSection}{Technical Strengths}

\begin{tabular}{ @{} >{\bfseries}l @{\hspace{6ex}} l }
Computer Languages & MATLAB, C++, C, System Verilog, Perl \\
\end{tabular}

\end{rSection}

\begin{rSection}{PERSONAL ACHIEVEMENTS}
\item[$\cdot$] Awarded Infineon Scholarship for academic excellence in 3rd year at RVCE, Bangalore in the year 2011.
\item[$\cdot$] Stood 1st academically in Electronics and Communication Engineering department, RVCE in the 1st and 3rd year. Secured 8th rank overall.
\item[$\cdot$] Secured a rank of 259 among 120,000 students in a common entrance test for Engineering in Karnataka, India year 2011.
\item[$\cdot$] Secured National level 17th rank in Mathematica (National Level Maths Exam) in the year 2005.
\end{rSection}

\end{document}
